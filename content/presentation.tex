\documentclass[aspectratio=169, 12pt, mathserif]{beamer}

\usepackage{fontspec}
%\usepackage{microtype}     % microtype works only with pdflatex
\usepackage[ngerman]{babel} % polyglossia makes problems with beamer

%
% temporaray packages
%
\usepackage{blindtext}

\newfontfamily\museo{Museo Slab}
\newfontfamily\helvetica{Helvetica Neue 45 Light}   % Helvetica Sans makes problems as .ttf
\setbeamerfont{frametitle}{family=\museo}
\setbeamerfont{block title}{family=\museo}
\setbeamerfont{block body}{family=\helvetica}
\setbeamerfont{title}{family=\museo}
%
% color themes and styles
%
%\usetheme{Frankfurt}
\newcommand{\thesisTitle}{Generierung und Design einer Client-Bibliothek für einen RESTful Web Service am Beispiel der Spreadshirt-API}
\newcommand{\thesisUniversityDepartment}{Fakultät für Informatik, Mathematik \& Naturwissenschaften}
\newcommand{\thesisKeyword}{Codegeneration, \textsc{Rest}ful Web Service, Modeling, Client-Library, Spreadshirt-\textsc{Api}, Polyglot}

\title[Generierung einer Client-Bibliothek aus einer WebService Beschr.]{\thesisTitle{}}
\subtitle{Bachelorverteidigung}

\author[A. Linz]{Andreas Linz}
\institute[]{\thesisUniversity{} - \thesisUniversityDepartment}

\date{\today}

\subject{Informatik}
\keywords{\thesisKeywords{}}

\begin{document}
    \begin{frame}[plain]
        \titlepage
    \end{frame} 
    
    \section{Inhaltsverzeichnis}
    \begin{frame}{Inhaltsverzeichnis}
        \tableofcontents
    \end{frame}

    \section{Einführung}
    \begin{frame}{Einführung}
        \begin{block}{Blindtext}
            Dies hier ist ein Blindtext zum Testen von Textausgaben. Wer
diesen Text liest, ist selbst schuld. Der Text gibt lediglich den
Grauwert der Schrift an. Ist das wirklich so? Ist es gleichgültig,
ob ich schreibe: Dies ist ein Blindtext“ oder Huardest
”
”
gefburn“? Kjift -- mitnichten! Ein Blindtext bietet mir wichtige
Informationen. An ihm messe ich die Lesbarkeit einer Schrift,
ihre Anmutung, wie harmonisch die Figuren zueinander stehen
und prüfe, wie breit oder schmal sie läuft. Ein Blindtext sollte
möglichst viele verschiedene Buchstaben enthalten und in der
Originalsprache gesetzt sein.
        \end{block}
    \end{frame}

    \subsection{Beispiel}
    \begin{frame}{Beispiel}
        Test
        \begin{exampleblock}{Titel}
            \texttt{int main(void) \{}\\
            \texttt{printf("Hello!");}\\
            \texttt{\}}
        \end{exampleblock}
        \begin{alertblock}{Titel}
            \begin{equation}
            \left[
            {\bf X} + {\rm a} \ \geq\ 
            \underline{\hat a} \sum_i^N \lim_{x \rightarrow k} \delta C
            \right]
            \end{equation}
        \end{alertblock}
    \end{frame}

    \begin{frame}
        \begin{block}{More about nothing \ldots}
            \blindtext[1]
        \end{block}
    \end{frame}

    \section{Listen}
    \begin{frame}
        \begin{block}{eine Liste}
            \begin{enumerate}
                \item A
                \item B
            \end{enumerate}
        \end{block}

        \begin{block}{Beschreibung}
            \begin{description}
                \item[foo] An ihm messe ich die Lesbarkeit einer Schrift,
ihre Anmutung, wie harmonisch die Figuren zueinander stehen
und prüfe, wie breit oder schmal sie läuft.
                \item[bar] An ihm messe ich die Lesbarkeit einer Schrift,
ihre Anmutung, wie harmonisch die Figuren zueinander stehen
und prüfe, wie breit oder schmal sie läuft.
            \end{description}        
        \end{block}
    \end{frame}
\end{document}
