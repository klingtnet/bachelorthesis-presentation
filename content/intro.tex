\section{Einleitung}

\begin{frame}
    \begin{block}{Denk dran!}
        Interesse wecken\\
        motivierender Aufhaenger\\
        Botschaft in den Mittelpunkt stellen
    \end{block}
\end{frame}

\begin{frame}{Aufgabe}
    \begin{block}{Was?}
        Client-Bibliothek aus abstrakter Beschreibung eines RESTful Web Service erzeugen
    \end{block}

    \begin{block}{Warum?}
        \begin{itemize}
            \item Vereinheitlichung besthender Implementierungen
            \item Nutzung der API für externe Entwickler erleichtern
            \item Authentifizierung kapseln
        \end{itemize}
    \end{block}
\end{frame}

\subsection{Anforderungen}
\begin{frame}{Anforderungen}
    \begin{itemize}
        \item Austauschbarkeit der Zielsprache
        \item einfache Bedienbarkeit der Bibliothek
        \item gute Lesbarkeit des erzeugten Codes
        \item größtmögliche Typsicherheit des erzeugten Codes
        \item {\color{gray} hohe Testabdeckung}
        \item vollständige Generierung der Methoden aus der API-Beschreibung
    \end{itemize}
\end{frame}

\subsection{Spreadshirt}
\begin{frame}{Spreadshirt}
    \begin{itemize}
        \item führendes Unternehmen für \emph{personalisierte Bekleidung}
        \item \emph{Social-Commerce}
        \item Standorte in Europa \& Nordamerika, HQ in Leipzig
        \item $\approx$ 450 Mitarbeiter, 50 in der IT
        \item $4*10^5$ Spreadshirt-Shops mit $33*10^6$ Produkten
    \end{itemize}
\end{frame}
